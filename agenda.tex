\title{Computer Science Society (CSS) Annual Agenda}
\author{
		Quinn Perfetto \\
    With Assistance From The 2016/2017 CSS 
}
\date{\today}

\documentclass[12pt]{article}

\usepackage{ amssymb }
\usepackage{longtable}
\usepackage[T1]{fontenc}
\usepackage[none]{hyphenat}
\usepackage{graphicx}
\usepackage{hyperref}

\begin{document}
\maketitle

\tableofcontents

\pagebreak

\section{Purpose}
This agenda should serve as a guideline for the minimum required involvement of 
the CSS on an annual basis.

\section{Industry Preparation}
The CSS should provide, or enlist a computer science (CS) student outside of the CSS
to provide, industry preparation through the following mediums:

\subsection{Problem of the Week (POTW)}
Each week an interview inspired programming problem is posted to be solved by the
CS student body.  Problems should be designed to force students to learn important
concepts not studied in class.  This activity is important as it exposes students
to interview style programming problems in a very accessible way.  POTWs could
also be coupled with semi-regular presentations, dependant on the level of
participation, which explain in detail the concepts required to solve each problem.
\\ \\
An open source django web app has been created for the purpose of managing these POTWs,
and can be found at: \\
\url{https://github.com/Quinny/uWindsor-POTW-Leaderboard}
\\
Suggested POTW concepts include, but are not limited to:
\begin{itemize}
  \item Bitwise operations
  \item Union Finds
  \item Tries
  \item Socket I/O
  \item Bloomfilters
  \item Security (MySQL injection, buffer overflow, brute force, etc.)
  \item Graph algorithms (BFS, DFS, shortest path, etc.)
\end{itemize}

\subsection{Internship Mentor}
The CSS should elect a student to provide mentoring to the CS student body in
all matters related to securing internships.  Duties may include performing
mock interviews, resume critiquing, giving presentations, etc. 

\section{Monthly Events}
In addition to the events listed below, the CSS should strive to hold at least
one social event per month.  Suggestions for these events include bowling, movie
nights, video game competitions, group dinners, etc.

\subsection{September}

\subsubsection{CSS T-Shirts}
Every year the CSS designs and produces T-shirts for the purpose of being
distributed to first year students.  Historically, the left over T-shirts have
been distributed to the rest of the student body, but only after all first years
have received one.  The number of incoming first years can be found asking
the CS secretary Margaret.  Generally T-shirt size distributions are largely
skewed towards mediums.

\subsubsection{Orientation}
Every September the President of the CSS should speak at orientation and inform
the incoming students about the contents of this document.  The President should
also lead the students to the St. Denis centre.

\subsubsection{Promote ACM Programming Contest}

\subsection{October}

\subsubsection{Volunteer or Compete in ACM Programming Contest}

\subsubsection{Decorate The Java Lab For Halloween}
Sometime in the first week of October the Java Lab should be decorated
for Halloween. Decorations can be found in the CSS office, and also may be
purchased using CSS funds if needed.

\subsubsection{Halloween Movie Night}
The week of Halloween the CSS should hold a scary movie night.  Pizza/snacks
can be provided using CSS funds.

\subsection{December}

\subsubsection{Decorate The Java Lab For Christmas}

\subsection{January}

\subsubsection{Student Run Programming Contest}

\subsection{February}
\subsection{March}
\subsection{April}

\subsection{Organize Election For Next Year}

\subsection{May}


\end{document}
