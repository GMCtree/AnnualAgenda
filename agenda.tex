\title{Computer Science Society (CSS) Annual Agenda}
\author{
		Quinn Perfetto \\
    With Assistance From The 2016/2017 CSS
}
\date{\today}

\documentclass[12pt]{article}

\usepackage{ amssymb }
\usepackage{longtable}
\usepackage[T1]{fontenc}
\usepackage[none]{hyphenat}
\usepackage{graphicx}
\usepackage{hyperref}

\begin{document}
\maketitle

\tableofcontents

\pagebreak

\section{Purpose}
This agenda should serve as a guideline for the minimum required involvement of
the CSS on an annual basis.

\section{Industry Preparation}
The CSS should provide, or enlist a computer science (CS) student outside of the CSS
to provide, industry preparation through the following mediums:

\subsection{Problem of the Week (POTW)}
Each week an interview inspired programming problem is posted to be solved by the
CS student body.  Problems should be designed to force students to learn important
concepts not studied in class.  This activity is important as it exposes students
to interview style programming problems in a very accessible way.  POTWs could
also be coupled with semi-regular presentations, dependant on the level of
participation, which explain in detail the concepts required to solve each problem.
\\ \\
An open source django web app has been created for the purpose of managing these POTWs,
and can be found at: \\
\url{https://github.com/Quinny/uWindsor-POTW-Leaderboard}
\\
Suggested POTW concepts include, but are not limited to:
\begin{itemize}
  \item Bitwise operations
  \item Union Finds
  \item Tries
  \item Socket I/O
  \item Bloomfilters
  \item Security (MySQL injection, buffer overflow, brute force, etc.)
  \item Graph algorithms (BFS, DFS, shortest path, etc.)
\end{itemize}

\subsection{Internship Mentor}
The CSS should elect a student to provide mentoring to the CS student body in
all matters related to securing internships.  Duties may include performing
mock interviews, resume critiquing, giving presentations, etc.

\section{Monthly Events}
In addition to the events listed below, the CSS should strive to hold at least
one social event per month.  Suggestions for these events include bowling, movie
nights, video game competitions, group dinners, etc.

\subsection{September}

\subsubsection{CSS T-Shirts}
Every year the CSS designs and produces T-shirts for the purpose of being
distributed to first year students.  Historically, the left over T-shirts have
been distributed to the rest of the student body, but only after all first years
have received one.  The number of incoming first years can be found by asking
the CS secretary Margaret.  Generally T-shirt size distributions are largely
skewed towards mediums. In the past we have had success in ordering from
Graphix Plus and dealing with Kirk.

\subsubsection{Orientation}
Every September the President of the CSS should speak at orientation and inform
the incoming students about the contents of this document.  The President should
also lead the students to the St. Denis centre.

\subsubsection{Promote The ACM Programming Contest}
The CSS should actively promote participation in the ACM programming contest,
and optionally provide workshops to help prepare new students for contest style
programming.

\subsection{October}

\subsubsection{Volunteer or Compete in ACM Programming Contest}
The CSS members are encouraged to lead by example by competing in, or volunteering
to help orgainize the ACM contest.

\subsubsection{Decorate The Java Lab For Halloween}
Sometime in the first week of October the Java Lab should be decorated
for Halloween. Decorations can be found in the CSS office, and also may be
purchased using CSS funds if needed.

\subsubsection{Halloween Movie Night}
The week of Halloween the CSS should hold a scary movie night.  Pizza/snacks
can be provided using CSS funds.

\subsection{November}

\subsubsection{Begin Considerations for CS Games}
CS games is a yearly programming contest generally held somewhere in Quebec.
Historically, uWindsor has sent two teams to compete.  In the opening weeks of
November the CSS should begin thinking about specific organization details and
begin the team selection process.  In the past the CSS selected one team as the
"A" team with the intent on trying to actually win events, and another team as the
"B" team, with emphasis on personal growth and grooming future "A" team participants.

\subsubsection{Register For CS Games}
Registration for CS games opens at the end of November.

\subsection{December}

\subsubsection{Decorate The Java Lab For Christmas}
Decorations can be found in the CSS office, and the Christmas tree is generally
held by the techs (inquire with Steve).

\subsection{January}

\subsubsection{Student Run Programming Contest}
Each year the CSS should hold a student run programming contest in a similar
style to the ACM contest.  This contest is best held within the first two weeks
of the Winter semester so that students are not too busy with classes to
participate.  Quinn Perfetto and David Valleau have designed an environment for
conducting these contests, contact them for details.


\subsection{February}
\subsection{March}

\subsubsection{Pi Day}
The CSS and the Math department organize an event to celebrate Pi Day (March 14\textsuperscript{th}).
The event involves purchasing pies for the students to eat, as well as small games such as "Most Memorized
Digits of Pi".

\subsection{April}

\subsubsection{Organize Election For Next Year}
The current President should email the CRO of the UWSA and begin discussions
about organizing the elections for the following year.  A student who will not
be running should be put in charge of organizing the election in order to avoid
potential conflicts of interest.

\subsection{May}

\subsubsection{Changing Of The Guard}
The new CSS should transition into place in May.  Signing power is to be assigned
to the new President, VP Academic, and VP Finance.  A CSS credit card can also
be applied for at this time by reaching out to the VP Finance for the UWSA. The new
2\textsuperscript{nd}, 3\textsuperscript{rd}, and 4\textsuperscript{th} year representatives
should also introduce themselves via email or the Facebook group.


\end{document}
